%%%%%%%%%%%%%%%%%%%%%%%%%%%%%%%%%%%%%%%%%%%%%%%%%%%%%%%%%%%%%%%
%	
% Welcome to Overleaf --- just edit your LaTeX on the left,
% and we'll compile it for you on the right. If you open the
% 'Share' menu, you can invite other users to edit at the same
% time. See www.overleaf.com/learn for more info. Enjoy!
%
%%%%%%%%%%%%%%%%%%%%%%%%%%%%%%%%%%%%%%%%%%%%%%%%%%%%%%%%%%%%%%%

% Inbuilt themes in beamer
\documentclass{beamer}

%packages:
% \usepackage{tfrupee}
% \usepackage{amsmath}
% \usepackage{amssymb}
% \usepackage{gensymb}
% \usepackage{txfonts}

% \def\inputGnumericTable{}

% \usepackage[latin1]{inputenc}                                 
% \usepackage{color}                                            
% \usepackage{array}                                            
% \usepackage{longtable}                                        
% \usepackage{calc}                                             
% \usepackage{multirow}                                         
% \usepackage{hhline}                                           
% \usepackage{ifthen}
% \usepackage{caption} 
% \captionsetup[table]{skip=3pt}  
% \providecommand{\pr}[1]{\ensuremath{\Pr\left(#1\right)}}
% \providecommand{\cbrak}[1]{\ensuremath{\left\{#1\right\}}}
% %\renewcommand{\thefigure}{\arabic{table}}
% \renewcommand{\thetable}{\arabic{table}}      

\setbeamertemplate{caption}[numbered]{}

\usepackage{enumitem}
\usepackage{tfrupee}
\usepackage{amsmath}
\usepackage{amssymb}
\usepackage{gensymb}
\usepackage{graphicx}
\usepackage{txfonts}

\def\inputGnumericTable{}

\usepackage[latin1]{inputenc}                                 
\usepackage{color}    
\usepackage{textcomp, gensymb}         
\usepackage{array}                                            
\usepackage{longtable}                                        
\usepackage{calc}                                             
\usepackage{multirow}                                         
\usepackage{hhline}                             
\usepackage{mathtools}
\usepackage{ifthen}
\usepackage{caption} 
\providecommand{\pr}[1]{\ensuremath{\Pr\left(#1\right)}}
\providecommand{\cbrak}[1]{\ensuremath{\left\{#1\right\}}}
\renewcommand{\thefigure}{\arabic{table}}
\renewcommand{\thetable}{\arabic{table}}   
\providecommand{\brak}[1]{\ensuremath{\left(#1\right)}}

% Theme choice:
\usetheme{CambridgeUS}

% Title page details: 
\title{AI1110 \\ Assignment-5} 
\author{Pettugadi Pranav CS21BTECH11063}
\date{\today}
\logo{\large \LaTeX{}}


\begin{document}

% Title page frame
\begin{frame}
    \titlepage 
\end{frame}
\logo{}


% Outline frame
\begin{frame}{Outline}
    \tableofcontents
\end{frame}



\section{Question}
\begin{frame}{Question}
    \begin{block}{\textbf{Papoullis EX 7-1:} } 
       Given independent random variables $x_i$ with respective densities $f_i(x_i)$ , we form random variables 
       $y_k = x_1 + x_2 + x_3 +....+ x_k$. Then show that random variables $y_i$ are also independent            \end{block}
     
\end{frame}



\section{Solution}
\begin{frame}{Solution}
\begin{block}{}

The system \\
y_1=x_1\\
y_2=x_1+x_2\\
y_3=x_1+x_2+x_3\\
.\\.\\.\\.\\
y_n=x_1+x_2+x_3+.....+x_n\\
has a unique solution\\
    $x_k=y_k-y_k_-_1)$
\end{block}
        \end{frame}
        
        \begin{frame}
\begin{block}{}
its jacobian transformation is \\
%%\begin{align}
    J(x_1 x_2 ...x_n) &=$ \begin{vmatrix}
\frac{\partial g_1}{\partial x_1} & ..& \frac{\partial g_1}{\partial x_n}\\
.&....&.\\
.&....&.\\
\frac{\partial g_n}{\partial x_1} & ..& \frac{\partial g_n}{\partial x_n}\\
       \end{vmatrix}$\\
  %%     \end{align}
      =$\begin{vmatrix}
      1 & 0 &..& 0 & 0\\
      0 & 1 &..& 0 & 0\\
      . & . &..& . & .\\
      0 & 0 &..& 1 & 0\\
      0 & 0 &..& 0 & 1\\
      \end{vmatrix}$
      =$1$
\end{block}
            \end{frame}
            
           \begin{frame}
           \begin{block}{}
                      then the joint density of y_k is
                      \end{block}
           \begin{align}
               f(y_1 y_2 .. y_n) &=\frac{f(x_1 x_2..x_n)}{J}\\
               &=
               f(x_1 x_2..x_n)\\
                         &=
               f(x_1)f(x_2)...f(x_n)\\
               &=
               f(y_2-y_1)f(y_3-y_2)...f(y_n-y_n_-_1)
                 \end{align}
 It follows that any subset of the set $x_i$ is a set of independent random variables.
 
 $\{y_1 \le y_1\}=\{x_1 \in A_1\},.., \{y_n \le y_n\}=\{x_n \in A_n\}$\\
 \begin{align}
 f(y_1 \le y_1\cap y_2 \le y_2..\cap y_n \le y_n)&=f(x_1\in A_1\cap x_2\in A_2..\cap x_n\in A_N)\\
 &=
 f(x_1\in A_1)f(x_2\in A_2)...f(x_n\in A_n)\\
 &=
 f(y_1\le y_1)f(y_2\le y_2)...f(y_n\le y_n)
 \end{align}
 \end{frame}
 
 \begin{frame}
 $\therefore$ If the random variables $x_i$ are independent, 
then the random variables
         $ y_1=g(x_1),y_2=g(x_2),..,y_n=g(x_n)$\\
  are also independent
 \end{frame}
\end{document}